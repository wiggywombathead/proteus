\section{Background}
\subsection{Relevant Material}
    To this end, there are a handful of modern resources for getting involved
    with operating systems development -- a particularly useful one at the time
    of first carrying out research for the project's proposal was
    \code{wiki.osdev.org}, which contains information about the creation of
    operating systems and acts as a community for hobbyist operating system
    developers. However, much of the focus is on the x86 platform and past
    providing a brief overview of the idiosyncrasies of the Raspberry Pi as well
    as the code to get a barebones kernel to boot, there is little material on
    the specifics to get core systems working on the platform.  Cambridge's
    \textit{Baking Pi} \cite{BakingPi} provides more help in this regard, with
    Alex Chadwick's comprehensive tutorials proving an invaluable resource for
    information such as accessing registers and peripherals specific to the
    Raspberry Pi. The project can, however, be much further extended to guide
    through the implementation of core operating system concepts such as memory
    management, the process model, inter-process communication, and filesystems.
    Another aspect in which this series of tutorials diverges with the goal of
    this project is the language in which it has been implemented -- while
    assembly is an undeniably language in which to be competent, it is not the
    most easily-understandable, in stark contrast to what this project hopes to
    achieve. The resource which aligns most tightly with the aim of this project
    is \cite{jsandler}, whose tutorials have served as an outline to how many
    key features of the project have been implemented.
    
    Finally, other notable resources which are in place to teach general
    operating systems development are Stanford's \textit{Pintos} \cite{Pintos}
    and Tanenbaum's MINIX operating system \cite{MINIX}; the former was written
    to accompany the university's CS140 Operating Systems course, while the
    latter is an illustrative operating system written alongside the book
    \textit{Operating Systems: Design and Implementation}, as a means of
    providing concrete examples of how operating system features are implemented
    in practice. Helin and Renberg's \textit{The Little Book About OS
    Development} \cite{littleosbook} also serves as a guide to writing one's own
    operating system. The only drawback to these three is their focus on the x86
    architecture, and while they are useful resources it is in concept only,
    given the gap which was found to quickly form from focusing on a different
    processor.

\subsection{Why is this project worthwhile?}
    The project is worthwhile firstly as it provides an accessible gateway into
    systems programming and operating systems development. Given the relative
    difficulty and additional effort required to get involved with this area of
    software development as opposed to others, for example by reading technical
    reference manuals and building an intimate knowledge of the hardware with
    which you are working before even starting, it therefore finds its use in
    easing this transition and making the learning curve associated with its
    involvement less intimidating and more approachable. In doing this, the
    project is also worthwhile in that it demystifies some of the key
    considerations that go into operating system implementations, not only in
    high-level concepts such as processes, but also the low-level with notions
    such as memory-mapped I/O and the processor's registers. In providing this
    opportunity to see theory in practice, it further opens up the opportunity
    for experimentation and invites practical self-learning, and hopefully
    clarifying why existing operating systems work the way they do.

    While there are similarities to be drawn between the aims of this project
    and those of the current background material, both looking to create a more
    accessible way in to operating systems development, this project addresses
    the gap that they leave unfilled by tackling a different architecture, as
    well as in approaching feature implementation in a more modular manner. Thus
    the project forms one more part in the ecosystem of introductory and
    instructional operating systems.

\subsection{Useful concepts}
    \noindent \textbf{Operating System} - a program that manages a computer's hardware
    that acts as an intermediary between the user and said hardware
    \cite{DinosaurOS}, providing an environment in which a user can execute
    programs conveniently and efficiently. \\

    \noindent \textbf{Kernel} - the one program running at all times throughout
    an operating system's execution, the kernel is often tasked with managing
    the most vital/recurring tasks. Other key types of programs include system
    programs and application programs. \\

    \noindent \textbf{Freestanding environment} - typical programs are written
    to run in a hosted environment, meaning they has access to a C standard
    library and other useful runtime features. Conversely, a freestanding
    environment is one which uses no such pre-supplied standard, meaning a
    bespoke one must be supplied. Any functions we wish to use as part of the
    operating system we must define ourselves. \\

    \noindent \textbf{Cross-compiler} - a regular compiler will generate
    machine-code which is specific to that on which the code has been compiled.
    By contrast, a  cross-compiler allows us to write code on any machine and
    compile it for our target architecture (the architecture on which we design
    our code to run). \\

    \noindent \textbf{Linker} - responsible for linking all object (\code{.o})
    files generated by a compiler/assembler into a single executable (or
    `library file', see \cite{StaticLibrary}). It also defines the entry point
    and the location of the various sections (detailed in Section
    \ref{list:LinkerSections}) in the final ELF file.\\

    \noindent \textbf{Exception} - an event triggered when something exceptional
    happens during normal execution, for example hardware providing the CPU with
    data, a privileged action, or bad instruction. \\

    \noindent \textbf{Process} - a program that has been loaded into memory and is
    executing. \\

    \noindent \textbf{Context Switch} - the act of saving the currently
    executing process into memory (\textbf{state save}) followed by loading the
    saved state of a different process (\textbf{state restore}). It allows the
    CPU to switch from executing one process to executing another. \\

    \noindent \textbf{Concurrency} - the ability for multiple processes to make
    progress seemingly simultaneously, as a result of being rapidly brought into
    and out of memory with respect to a policy enforced by some scheduling
    algorithm. \\

    \noindent \textbf{Synchronisation} - the prevention of \textbf{race
    conditions}: when the outcome of two concurrently executing processes
    depends on the order in which data access took place. \\

    \noindent \textbf{Inter-process Communication} - the mechanism by which
    processes may exchange data and information. \\
        
    \noindent \textbf{Peripheral} - a device with a specific memory address
    which it may write data to and read data from. All peripherals may be
    described by an offset from some base address, covered in Section
    \ref{sec:GPIO}. \\

    \noindent\textbf{Note:} Throughout this report the units kiB and MiB are
    used (mostly in the context of memory) to denote $2^{10} = 1024$ bytes and
    $(2^{10})^2=1,048,576$ bytes, respectively.  This is to ensure clarity and
    avoid confusion with their SI-prefixed counterparts, namely kB and MB, which
    instead correspond to $10^3$ and $10^6$ bytes.
