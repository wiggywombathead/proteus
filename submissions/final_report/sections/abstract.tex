% 1. Purpose
% 2. Methodology/Project Design 
% 3. Findings/Contribution
% 4. Research Limitations/Future work
% 5. Practical Implications/Conclusions

% No jargon - understand all terms without additional reading
\begin{abstract}
    This project presents an educational operating system for the Raspberry Pi 1
    Model B+, written for the purpose of demystifying aspects of operating
    systems development for the hobbyist programmer, especially with regards to
    low-level systems programming and core features to a computer's execution.
    It provides a small multiprocessing kernel developed with the aim of clarity
    of understanding and ease of extension. Together with a simple interface to
    further configure the system at compile-time to modify its approaches to
    process scheduling, it aims to promote a practical approach to operating
    systems education. This interface can be easily extended to accommodate
    different models of inter-process communication. Future work should be aimed
    at increasing the operating system's usability in a real-world context, in
    particular with regards to user input and access to permanent storage, of
    which it has none. While it is limited in this sense, it provides a simple
    and open testbed on which to both study how key operating system concepts
    are implemented in practice, as well as to invite the addition of new
    features.
\end{abstract}
