\section{Testing and Issues}
    As mentioned in Section \ref{sec:design_testing}, the project was tested
    manually and consistently throughout its development. Detailed here are some
    particular issues faced during the various testing phases, and how they were
    dealt with.

\subsection{Loading onto real hardware}
    Vital to ensuring that the system truly works is testing it on real
    hardware. As covered in Section \ref{sec:booting}, the Pi requires firmware
    containing its three bootloaders in order to bring a kernel image into
    memory and begin execution. The most simple way to acquire this was to
    download an existing operating system for the Pi, since this would already
    contain all the files required, and simply replace the \code{kernel.img}
    with that generated during compilation. We use the toolchain utility
    \code{arm-none-eabi-objcopy} to copy the generated ELF file to a raw binary
    image. The image is then copied to the SD card to replace that of Raspbian,
    meaning the project's operating system will run in its place.

\subsubsection{Integer division on the ARM CPU}
    When compiling, GCC raised the error of an unrecognised symbol within the
    compiled object files, \code{\_\_aeabi\_uidivmod}, and after debugging it
    was discovered to have originated from use of the modulo operator. The cause
    of this is the fact that the ARM family of CPUs does not implement a native
    integer division instruction \cite{ARM_udiv}, and thus requires a custom
    implementation. A solution, at least early on, was to link against the
    \code{-lgcc} flag, as GCC provides its own implementation in this library.
    However, since an important, if implicit, personal requirement of the
    project is that it is self-sufficient wherever possible, it was decided to
    write the long division algorithm in assembly. The \code{\_\_aeabi\_*div*}
    family of functions require that the quotient of the division is stored in
    \code{r0} by the time the function returns, and that the remainder is in
    \code{r1}. Listing \ref{lst:uidiv} shows the implementation of
    \code{\_\_aeabi\_divmod(numerator, denominator)}, which allows for unsigned
    integer division and modulo operations. The implementation for
    \code{\_\_aeabi\_uidiv()} simply calls \code{\_\_aeabi\_uidivmod()}, as both
    have the same requirements regarding the contents of registers 0 and 1.
        
    \lstset{language=c}
    \begin{lstlisting}[caption={Implementation of unsigned integer division and
    modulo in ARM assembly},captionpos=b,label={lst:uidiv}]
__aeabi_uidivmod:
    mov r2, #0  // quotient
    loop:
        cmp r0, r1
        blo return
        sub r0, r0, r1
        add r2, r2, #1
        b loop
    return:
        mov r1, r0
        mov r0, r2
        bx lr
    \end{lstlisting}

\subsection{Interacting with the Memory Management Unit}
    Another issue during testing was the compiler warning of the deprecation of
    the \code{swp} instruction in ARMv6 and higher, as used in the
    implementation of the atomic swap operation in Listing \ref{lst:Lock}. It
    was discovered that the Load Register Exclusive and Store Register Exclusive
    instructions, \code{ldrex} and \code{strex} respectively, were preferred for
    this operation. It was further discovered \cite{CacheEnable} that these
    instructions were available only with both caches and the MMU enabled. While
    the project does not work when using these exclusive loads and stores,
    causing a data abort when attempting to do so, it does succeed in
    initialising the MMU, allowing the programmer to define their own map from
    virtual to physical memory using the \code{mmu\_section()} interface. This
    is done through a sequence of \code{mrc} and \code{mcr} instructions, the
    exact details of which were found at \cite[pg.~3-14]{TRM} -- it includes,
    for example, first invalidating the caches and Translation Lookaside Buffer
    (TLB), followed by setting the Translation Table base address. The memory
    map is then performed by setting various flags, which control whether
    portions of memory are buffered and/or cached, and defining the physical
    addresses to which virtual addresses should map. Much of this code is,
    however, accredited to David Welch, having been largely lifted from
    \cite{dwelch67}. Not being a primary objective of the project, and having
    been initially with the intention simply to make use of the non-deprecated
    atomic swap instructions, the use of this code was judged to be permissible.
    The option to use the MMU is, however, available for use if desired.

\subsection{ACT debugging}
    Lastly, the most notable and rewarding phase of testing was performed via
    the GPIO -- after loading onto real hardware, but before getting framebuffer
    initialisation working, there was no means of getting visual feedback from
    the real-world (i.e. non-emulated) system, other than by blinking the ACT
    LED. Initially this was done in assembly, however as this was cumbersome to
    control the number of times to blink the LED, an interface was soon
    developed in C to achieve the same goal. Furthermore, it uses the same
    interface developed for initialising the UART, namely that of
    \code{mmio\_read()} and \code{mmio\_write()}, to write to the correct
    GPIO registers to firstly initialise the LED, and then turn it on and off.
    Turning the ACT LED on, for example, is given by the simple interface in
    Listing \ref{lst:ACT_on}, requiring only one write to the General Purpose
    Set Register, which on the Pi 1 is address \code{0x20200020}.

    \lstset{language=c}
    \begin{lstlisting}[caption={Interface for controlling the
    ACT},captionpos=b,label={lst:ACT_on}]
void act_on(void) {
    mmio_write(ACT_GPSET, 1 << ACT_GPBIT);
}
    \end{lstlisting}

    The ACT was initially used to highlight when the code was reaching certain
    points in the kernel, and placing various calls to \code{act\_on()} and
    \code{act\_off()} around blocks of code could allow for errors, which would
    otherwise be hard to detect, be located relatively easily. Its utility
    further increased when the system timer was implemented -- this allowed for
    programming of the ACT to blink in different intervals, and lends itself to
    defining certain patterns of blinks to signify various states that the code
    is in (similarly to motherboard beep codes, for example).

    Soon after HDMI output worked correctly, focus was paid on implementing
    helpful debugging functions, in particular \code{printf()}. Several other
    functions from the C standard library are implemented in the files
    \code{stdio.h}, \code{stdlib.h}, and \code{string.h}. Recall that since we
    are only in a freestanding environment, the project implements almost the
    entire C standard library itself -- the only files it initially has access
    to are \code{<float.h>}, \code{<iso646.h>}, \code{<limits.h>},
    \code{<stdalign.h>}, \code{<stdarg.h>}, \code{<stdbool.h>},
    \code{<stddef.h>}, \code{<stdint.h>}, and \code{<stdnoreturn.h>}. ACT
    debugging still found its use later, but where possible the more efficient
    method of \code{printf()} was preferred.
