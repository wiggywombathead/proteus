\documentclass[10pt,a4paper]{article}

\bibliographystyle{ieeetr}

\usepackage[margin=1in]{geometry}
\usepackage{graphicx}
\usepackage{subfig}
\usepackage{amsmath}
\usepackage{url}
\usepackage{diagbox}

\title{A moduler kernel for the Raspberry Pi: Project Specification}

\begin{document}

\maketitle

\begin{center}
    Thomas Archbold \\
    1602581 \\
    University of Warwick \\
\end{center}

\section{Background}
In most operating systems, many design decisions are made in order to keep
things simple for the user. Compromises are made to keep most of the technical
details hidden, and in most cases this is an appropriate approach: needlessly
offering more choices for low-level tasks that are usually handled by the
operating system would only serve to confuse. Indeed, in the worst case, it
could compromise the stability and security of the system by allowing more
opportunities for errors to be introduced.  This more insulated approach does
mean, however, that the user never really knows what is going on ``under the
hood'', and indeed whether greater performance can be achieved by making
\textit{different} fundamental decisions.  Furthermore, a number of operating
systems exist for the Raspberry Pi, some focusing on ease-of-installation,
others on Internet of Things integration, but none exist to serve as a testbed
for low-level design decisions. Thus this project aims to fill this gap for the
operating systems enthusiast, one who wishes to test for themselves the
different approaches to such vital approaches to CPU scheduling, disk
scheduling, interprocess communication, and filesystem structures. It will give
the user the ability to dynamically change the fundamental ways in which their
machine operates by loading different modules to handle different tasks, without
the need to reboot, enabling for a more flexible operating system where
performance can measured and tweaked at any point.

\section{Objectives}
This project will attempt to create a small but functional operating system for
the Raspberry Pi 2 Model B, that is capable of loading different modules to
tackle CPU scheduling, disk scheduling, interprocess communication, and
filesystems in different ways. Crucially, the kernel must be bootable on real
hardware, must be able to take input from a physical keyboard, and display any
relevant/necessary output on a real screen. Specifically, it must be able to run
multiple processes at once through utilising a CPU scheduler; interact with a
hard disk drive and disk scheduler for mass storage; create, read, update, and
delete files and directories using a custom filesystem; and be able to load
modules dynamically for CPU and disk scheduling, interprocess communication, and
a filesystem on demand.

\subsection{Core aims}
The operating system must be able to:
\begin{itemize}
    \item boot successfully from either GRUB or a custom bootloader
    \item take keyboard input via the Pi's USB port
    \item send output to a screen using the Pi's HDMI or serial port
    \item implement a small shell/comand interpreter
    \item exhibit some degree of multiprogramming
    \item interface with a hard drive for mass storage
    \item read from and write to a custom filesystem
    \item implement an interface for dynamically switching modules for various
        tasks
    \item run stably throughout its lifetime
    \item shutdown safely
\end{itemize}

\subsubsection{Loadable modules}
The following data structures and algorithms will be implemented as loadable
modules, which may be switched to on-the-fly:
\begin{itemize}
    \item CPU Scheduling \cite{DinosaurCPU}:
        \begin{itemize}
            \item First Come First Served
            \item Round Robin
            \item Shortest Job First
            \item Shortest Remaining Time First
            \item Priority Scheduling (preemptive and non-preemptive)
            \item Lottery Scheduling
        \end{itemize}
    \item Disk Scheduling \cite{DinosaurDisk}:
        \begin{itemize}
            \item First Come First Served
            \item Shortest Seek Time First
            \item SCAN and C-SCAN (elevator algorithm)
            \item LOOK and C-LOOK
        \end{itemize}
    \item Interprocess Communication
        \begin{itemize}
            \item Message passing
            \item Shared memory
        \end{itemize}
    \item Filesystem
        \begin{itemize}
            \item persistent
            \item load-on-request
        \end{itemize}
\end{itemize}

\subsection{Stretch goals}
Some stretch goals which would not be entirely necessary for the success of the
project, but should be implemented to show understanding of more complex
structures, would be primarily some more intricate scheduling algorithms,
including the following:
\begin{itemize}
    \item Multilevel Queue and Multilevel Feedback Queue
    \item Completely Fair Scheduler
    \item $O(n)$ Scheduler
    \item $O(1)$ Scheduler
    \item Staircase Deadline Scheduler
    \item Multiple Queue Skiplist Scheduler, MuQSS (a reimplementation of Con
        Kolivas' Brain Fuck Scheduler)
\end{itemize}

In order to give the operating system more purpose and to increase
usability, a simple text editor would be useful and should be implemented.
Looking ahead much further into areas for development, it would be useful to
implement a C compiler, so that the system is fully functional and on par with
modern operating systems.

\subsection{Further extensions}
Beyond this, the project could implement more or all  of the complex CPU
schedulers from the list above. Any further extensions would aim to increase the
operating system's usability, and bring it more in line to what we expect from
an operating system. As the system will be built with modularity as a key focus,
this should aid in the development of additional functionality, and leave the
option open for features such as networking and security. The timespan for these
additions is likely to extend past the project deadline, however.

\section{Methodology}
The methodology best suited to the project will be a mix between a plan-driven
and agile approach; the basic requirements of the system will not change over
the course of the project, and furthermore there will be a rigid structure with
regards to dependencies that the project is likely to abide by (for example, the
system will have to boot before implementing memory management before
implementing scheduling algorithms). Therefore, the early stages of the project
will benefit from a plan driven approach, most likely an Incremental one to
allow for some choice in what to implement, as opposed to the more restrictive
structure of a Waterfall methodology. After the foundations have been
implemented successfully, the project is likely to open up and take a more agile
approach; Scrum cycles are likely to be useful dedicating a large portion of
concentration implementing one feature, or fixing specific bugs, at one time, in
an incremental manner.

Throughout the project, weekly meetings will be held with the supervisor in
order to discuss any current problems and talk through approaches to solutions
(especially for the more complex ones), the overall progress of the project, as
well as the direction in which it is headed. It would also be at this time that
progress is compared with the timetable, and any notes and adjustments are made
dynamically in order to fully stay on top of the work.

\section{Testing}
The project will be tested in an incremental manner. Especially to begin with,
it is vital that some systems operate correctly before moving on and developing
other areas. As the project progresses and its complexity increases, unit tests
will be written to systematically cover all, or at least most, likely paths of
execution, and to account for each of these. The most fundamental requirement to
fulfill while testing the solution will be stability, that is to say, whether
the system is able to safely switch between different modules and continue
operation. Of course, the solution must also be correct: the user must be able
to switch dynamically between the different modules, and the system must react
accordingly. There must be a way to verify that the system is indeed operating
in the way that is expected from the user, and again, unit tests and
verification software must be produced to ensure this.

\section{Timetable}

\section{Technologies}
The following technologies will be used by the project:
\begin{itemize}
    \item Git - version control
    \item Github - to access the project from multiple sources, as well as to
        back it up
    \item C - the language in which most of the operating system will be
        implemented
    \item ARM assembly - used when C is unavailable/inappropriate
        \cite{CannotDoC}
    \item GCC cross compiler for ARM EABI - for cross compiling for the target
        processor, the Cortex-A7 \cite{CrossCompilation}
    \item QEMU - for emulating the Pi to allow quicker and safer testing
        \cite{qemu}
    \item Make - used to speed up the build process
\end{itemize}

\section{Resources}
The following documentation will be used throughout for reference to the
architecture of the Cortex-A7 processor and its intruction set:
        \begin{itemize}
            \item Cortex-A7 MPCore Technical Reference Manual
            \item ARM Cortex-A Series Programmer's Guide
            \item Broadcom BCM2835 ARM Peripherals Manual
        \end{itemize}

\section{Legal, social, ethical, and professional  considerations}
All software used to build the project is available to use under the GNU Public
License. Throughout the project's development, some testing will be required
from people other than the creator, to gain feedback especially with regards to
usability; these people are likely to be friends and colleagues, hence the
social, ethical, and professional issues are insignificant.

\bibliography{bibliography}

\end{document}
