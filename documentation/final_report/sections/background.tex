\section{Background}
\subsection{Motivation}
    The main goal of this project is to provide an operating system for the
    Raspberry Pi which serves as a tool for operating systems education. A key
    aspect to this is its encouragement to learn-by-doing, and as such the
    project aims to be both configurable at compile-time as well as open to
    further extension. While it would largely be possible to implement in a
    solely emulated environment using a virtual machine, an important aspect is
    that the project runs on real hardware, namely a Raspberry Pi 1 Model B+
    using an HDMI connection for display and a MicroSD card for permanent
    storage, in order to provide the additional interaction with real-world
    considerations for operating system development. In order to provide the
    possibility of extension, the design of generic interfaces for such tasks as
    CPU scheduling, inter-process, and filesystems was also a key goal, as well
    as providing a simple interface to manage this.

    An operating system draws together aspects from all over the field of
    computer science, whose development requires intimate knowledge of low-level
    concepts such as the computer's organisation and architecture, up to an
    understanding for the more abstract in designing how processes communicate
    or implementing filesystems. Thus, in additional to the functional aims of
    the project, the one of key personal motivations behind the project was to
    gain experience in low-level systems programming and interacting with
    real-world hardware, as well as in creating an entirely self-contained and
    useful piece of software.

\subsection{Relevant Material}
    To this end, there are a handful of modern resources for getting involved
    with operating systems development -- a particularly useful one at the time
    of first carrying out research for the project's proposal was
    \code{wiki.osdev.org}, which contains information about the creation of
    operating systems and acts as a community for hobbyist operating system
    developers. However, much of the focus is on the x86 platform and past
    providing a brief overview of the idiosyncrasies of the Raspberry Pi as well
    as the code to get a barebones kernel to boot, there is little material on
    the specifics required to get core systems working on the platform.
    Cambridge's \textit{Baking Pi} \cite{BakingPi} provides more help in this
    regard, with Alex Chadwick's comprehensive tutorials proving an invaluable
    resource for information such as accessing registers and peripherals
    specific to the Raspberry Pi. The project can, however, be much further
    extended to guide through the implementation of core operating system
    concepts such as memory management, the process model, interprocess
    communication, and filesystems. Another aspect in which this series of
    tutorials diverges with the goal of this project is the language in which it
    has been implemented -- while assembly is an undeniably language in which to
    be competent, it is not the most easily-understandable, in stark contrast to
    what this project hopes to achieve. The resource which aligns most tightly
    with the aim of this project is \cite{jsandler}, whose tutorials have served
    as an outline to how many key features of the project have been implemented.
    
    Finally, other notable resources which are in place to teach general
    operating systems development are Stanford's \textit{Pintos} \cite{Pintos}
    and Tanenbaum's MINIX operating system; the former was written to accompany
    the university's CS140 Operating Systems course, while the latter is an
    illustrative operating system written alongside the book \textit{Operating
    Systems: Design and Implementation} \cite{MINIX}, showing how features are
    implemented in practice. Helin and Renberg's \textit{The Little Book About
    OS Development} \cite{littleosbook} also serves as a guide to writing one's
    own operating system. The only drawback to these three is their focus on the
    x86 architecture, and while they are useful resources it is in concept only,
    given the gap which was found to quickly form from focusing on a different
    processor.

\subsection{Why is this project worthwhile?}
%TODO
