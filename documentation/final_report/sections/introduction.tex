\section*{Introduction}
    Operating systems are some of the most pervasive pieces of software, but due
    to the nature of what they set out to achieve they are also some of the most
    complex, and are often impenetrable to understand without specialist
    knowledge. While widespread access to a personal computer is nothing new,
    the introduction of the Raspberry Pi in recent years has made experimenting
    with computers much more affordable and hence readily accessible, inviting
    tinkering at all levels with a reduced fear of economic forfeit. The Pi
    therefore provides the ideal platform for operating systems education --
    hobbyist developers looking to get involved in writing such systems have
    access to a standardised set of hardware that is inexpensive both to buy and
    to replace, if and when things go wrong. Of course, there are numerous
    official operating systems available for the Raspberry Pi, each addressing
    different issues such as ease-of-installation, Internet of Things
    integration, or classroom management \cite{OSes}, and many more unofficial ones,
    however there is little in the way of one which can be used to learn about
    the operating system itself. This project addresses this gap by providing a
    configurable, educational kernel for the Raspberry Pi 1 Model B+, with a
    focus on presenting code that is intuitive and simple to understand and
    providing clear interfaces for ease of extensibility of the system.

